%-*- Mode:LaTeX; -*-      
\documentclass[11pt]{article}
\usepackage{vmargin}		% Force narrower margins
\setpapersize{USletter}
\setmarginsrb{1.0in}{1.0in}{1.0in}{0.6in}{0pt}{0pt}{0pt}{0.4in}
\setlength{\parskip}{.1in}  % removed space between paragraphs
\setlength{\parindent}{0in}

\usepackage{epsfig}
\usepackage{graphicx}

\begin{document}

\large
\begin{center}
{\bf CS-5340/6340, Written Assignment \#1} \\
{\bf DUE: Wednesday, Sept. 9, 2015 by 11:00pm}
\end{center}
\normalsize

\begin{enumerate}  

\item (40 pts, 1/2 pt per word) For each sentence below, label each word with its
  correct part-of-speech (POS) tag based upon the word's use in the sentence.
  Punctuation should be ignored.

  Choose from the following list of part-of-speech tags: {\bf
    adjective ({\sc adj}), adverb ({\sc adv}), article ({\sc art}),
    conjunction ({\sc conj}), gerund ({\sc ger}), infinitive ``to''
    ({\sc inf}), modal ({\sc mod}), noun ({\sc noun}), particle ({\sc
      part}), preposition ({\sc prep}), personal pronoun (not
    possessive) ({\sc perpro}), possessive pronoun ({\sc posspro}),
    relative pronoun ({\sc relpro}), verb ({\sc verb})}.

For infinitive verb phrase constructions, label ``to'' as {\sc inf}
and the verb itself as {\sc verb}. 

NOTE: An easy way to show your part-of-speech tags is to append a slash and
POS tag after each word. For example: ``Natural/{\sc adj}
language/{\sc noun} is/{\sc verb} fun/{\sc adj}.''

\begin{enumerate}

\item Dan slid sideways down the hill and broke out in laughter. \\


Answer : Dan/{\sc noun} slid/{\sc verb} sideways/{\sc adv} down/{\sc prep} the/{\sc art} hill/{\sc noun} and/{\sc conj} broke/{\sc verb} out/{\sc adv} in/{\sc prep} laughter/{\sc noun}.\\


\item Children may like candy but eating sugary foods is unhealthy. \\

Answer: Children/{\sc noun} may/{\sc mod} like/{\sc prep} candy/{\sc noun} but/{\sc conj} eating/{\sc ger} sugary/{\sc adj} foods/{\sc noun} is/{\sc verb} unhealthy/{\sc adj}.\\
\item My brother did not plan to  take sleeping pills although he got
  no sleep yesterday.  \\
  
  Answer : My/{\sc posspro} brother/{\sc noun} did/{\sc verb} not/{\sc adv} plan/{\sc noun} to/{\sc inf}  take/{\sc verb} sleeping/{\sc ger} pills/{\sc noun} although/{\sc conj} he/{\sc perpro} got/{\sc verb} no/{\sc adv} sleep/{\sc verb} yesterday/{\sc adv}.\\

\item He often snores like an aardvark.  \\

Answer : He/{\sc perpro} often/{\sc adv} snores/{\sc verb} like/{\sc prep} an/{\sc art} aardvark/{\sc noun}.\\

\item Mary ran inside after rain began to fall.  \\

Answer : Mary/{\sc noun} ran/{\sc verb} inside/{\sc adv} after/{\sc prep} rain/{\sc noun} began/{\sc verb} to/{\sc inf} fall/{\sc verb}.  \\

\item The armed man took off when police showed up.  \\

 Answer : The/{\sc art} armed/{\sc adj} man/{\sc noun} took/{\sc verb} off/{\sc part} when/{\sc conj} police/{\sc noun} showed/{\sc verb} up/{\sc prep}.  \\

\item The kittens sleeping in her lap are very young.   \\

Answer : The/{\sc art} kittens/{\sc noun} sleeping/{\sc ger} in/{\sc prep} her/{\sc posspro} lap/{\sc noun} are/{\sc verb} very/{\sc adv} young/{\sc adj}.   \\

\item She just completed a singing competition, which could make her a star.  \\

Answer : She/{\sc perpro} just/{\sc adv} completed/{\sc verb} a/{\sc art} singing/{\sc ger} competition/{\sc noun}, which/{\sc relpro} could/{\sc mod} make/{\sc verb} her/{\sc posspro} a/{\sc art} star/{\sc noun}.  \\

\end{enumerate}



\newpage
\item (20 pts) For each sentence below, indicate whether the main verb
  appears in an {\it intransitive} construction, a {\it transitive}
  construction, or a {\it ditransitive} construction. Only give the
  answer {\it transitive} if the   usage is \underline{not} {\it ditransitive}.

\begin{enumerate}

\item The dog barked at the cat. \\

Answer: Main Verb - barked , transitive\\

\item The man fed the squirrels peanuts.  \\

Answer: Main Verb - fed , ditransitive \\

\item Susan slept for ten hours. \\

Answer: Main Verb - slept , transitive\\

\item George broke the window with his fist. \\

Answer : Main Verb - broke, ditransitive \\

\item Mary loaned her neighbor a bicycle for a week.  \\

Answer: Main Verb - loaned, ditransitive\\

\item Ted donated five hundred dollars to his favorite charity. \\

Answer: Main Verb - donated, ditransitive\\

\item Wilma married Fred in a rock quarry.  \\

Answer: Main Verb - married, ditransitive\\

\item Sam bought flowers for his mom. \\

Answer: Main Verb - bought, transitive\\

\item The cat frequently sits on the front porch. \\

Answer: Main Verb - sits, transitive\\

\item She gave a raise to her best employee for his great work.  \\

Answer: Main Verb - gave, ditransitive\\

\end{enumerate}


\newpage
\item (20 pts) For each sentence below, indicate whether the main verb
  appears in an {\it active} voice verb phrase or a {\it passive}
  voice verb phrase. 

\begin{enumerate}

\item Dr. Seuss has written many books. \\

Answer: Main Verb - written, {\it active}\\

\item Tim will be organizing a charity event. \\

Answer: Main Verb - organizing, {\it active}\\

\item Cathy has been hired by IBM. \\

Answer: Main Verb - hired, {\it passive}\\

\item Walter will be evaluated for a raise in October.  \\

Answer: Main Verb - evaluated, {\it passive}\\

\item Tropical storm Fred has strengthened into a hurricane. \\

\item The battle is being fought on several fronts.  \\

Answer:{\it passive}\\

\item The cougar was hiding in a bush. \\

Answer : {\it active}\\

\item The dog has been taught twenty difficult tricks.  \\

Answer: {\it passive}\\

\item She will have achieved a record in gymnastics. \\

Answer: {\it active}\\

\item He should have won the award.  \\

Answer: {\it active}

\end{enumerate}


\newpage
\item (20 pts) Consider the following morphology rules and dictionary:

\begin{center}
\begin{tabular}{|l|l|l|l|l|l|l|} \hline
~ & \textbf{Suffix} & \textbf{Prefix} & \textbf{Replacement} & \textbf{POS of} & \textbf{POS of } \\
~ & ~ & ~ & \textbf{Chars} & \textbf{root word} & \textbf{derived word} \\ \hline
{\it Rule \#1} & s & - & - & NOUN & NOUN \\
{\it Rule \#2} &s & - & - & VERB & VERB \\
{\it Rule \#3} &er & - & - & VERB & NOUN \\
{\it Rule \#4} &- & re & - & VERB & VERB \\
{\it Rule \#5} &- & anti & - & NOUN & ADJECTIVE \\ \hline
\end{tabular}
\end{center}

\begin{center}
\begin{tabular}{|l|l|} \hline
\multicolumn{2}{|c|}{\bf Dictionary} \\ \hline
{\bf Word} & {\bf Part-of-Speech} \\ \hline
seizure & NOUN \\
form & VERB \\ \hline
\end{tabular}
\end{center}

For each word below, indicate whether that word {\it CAN} or {\it CANNOT} be
successfully derived as having the specified part-of-speech using the
morphology rules and dictionary above:

\begin{enumerate}
\item antiseizure ADJECTIVE \\

Answer : {\it CAN}\\

\item seizures VERB \\

Answer: {\it CANNOT}\\

\item antiseizures NOUN \\

Answer : {\it CANNOT}\\

\item antiseizures ADJECTIVE \\

Answer: {\it CAN}\\

\item reforms NOUN \\

Answer: {\it CANNOT}\\

\item reforms VERB \\

Answer: {\it CAN}\\

\item antireform ADJECTIVE \\

Answer: {\it CANNOT}\\

\item rereform VERB \\

Answer: {\it  CAN}\\

\item reformer NOUN \\

Answer: {\it CAN}\\

\item reformers VERB \\

Answer: {\it CANNOT}\\
 
\end{enumerate}



\newpage
\underline{\textbf{Question \#5 is for CS-6340 students ONLY!}}  \\

\item (15 pts) Consider the following five subcategorization frames:

\begin{center}
\begin{tabular}{|l|} \hline
NP  \\
NP NP \\
PP(against) \\
PP(from) PP(to) \\
VP(to) \\ \hline
\end{tabular}
\end{center}

For each verb below, list \underline{ALL} of the subcategorization
frames in the list above that should be associated with the verb. 
If a verb should not have ANY of these
subcategorization frames, then give the answer {\it NONE}.

\noindent
HINT: most of the verbs should have 1 or 2 of the subcategorization
frames in the list above.  \\

\begin{enumerate}

\item snore \\

Answer: {\it NONE}\\

\item drive  \\

Answer :  PP(from) PP(to) , VP(to) \\

\item expect \\

Answer: NP\\

\item fight \\

Answer: NP NP e.g The battle of Waterloo was fought between Napoleon and the Duke of Wellington\\

\item sip \\

Answer: NP\\

\item sing \\

Answer: {\it NONE}\\

\item lean \\

Answer: PP(against)\\

\item smile \\

Answer: {\it NONE}\\

\item lend \\

Answer: NP, NP NP\\

\item increase \\

Answer : NP\\

\end{enumerate}

\end{enumerate}  % END OF WRITTEN QUESTIONS



\newpage
\hspace*{1.5in}  {\bf ELECTRONIC SUBMISSION INSTRUCTIONS \\
\hspace*{1.5in} (a.k.a. ``What to turn in and   how to do it'')} 

{\bf Your written assignment \underline{must} be in .pdf format.} Please do not turn in
.doc or .docx files ... convert them to .pdf format before submitting them!

To submit this assignment, the CADE provides a web-based
facility for electronic handin, which can be found here:
\begin{center}
https://webhandin.eng.utah.edu/
\end{center}
Or you can log in to any of the CADE machines and issue the command:
\begin{center}
handin cs5340 written1 $<$filename$>$
\end{center}
Please name your file: YourName-written1.pdf (e.g., EllenRiloff-written1.pdf)\\


\vspace*{.2in}
\hrule
HELPFUL HINT: you can get a listing of the files  that you've already
turned in via electronic submission by using the `handin' command
without giving it a filename. For  example:
\begin{center}
handin cs5340 written1
\end{center}
will list all of the files that you've turned in thus far. 
If you submit a new file with the same name as a previous file, the
new file will overwrite the old one.  


\end{document}

